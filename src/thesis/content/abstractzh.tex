%!TEX root = ../csuthesis_main.tex
% 设置中文摘要
\keywordscn{多无人机\quad 边缘计算\quad 任务分配\quad 航迹规划\quad 资源调度\quad 智能优化算法}
%\categorycn{TP391}
\begin{abstractzh}

近年来,随着无人机技术的不断发展,无人机在军事和民用领域都得到了广泛的应用。为了推广无人机在城市环境下的应用,本文针对无人机航迹规划及任务调度问题进行研究,在对该问题进行详细描述及分析的基础上,基于边缘计算架构将其分解为了任务计算资源调度问题、多无人机航迹规划及任务分配问题共两个子问题,分别建立了相应的数学模型,并通过设计基于ASA算法的静态任务计算资源调度算法、基于FD算法的动态任务计算资源调度算法、基于RRT*-Connect算法的静态场景下的无人机航迹规划算法、基于A*算法的动态场景下的无人机航迹规划算法以及基于TSAVN算法的多无人机任务分配算法实现对上述两个子问题的求解。最后,对于各个子问题分别设计了相应的仿真实验,验证了本文所提出的算法在求解考虑边缘计算的无人机航迹规划及任务调度各个子问题的有效性及优越性。

仿真实验结果表明:
% 任务计算资源调度问题
在任务计算资源调度问题中,ASA算法能够在考虑传输时延等情况下生成较优的计算资源调度方案,同时相较于其他对比算法,其生成的计算资源调度方案与算法的运行时间都有着显著的优越性和鲁棒性;而FD算法相较于其他对比算法,虽然其生成的计算资源调度方案较差,但其运行时间能够满足问题场景下所需要的实时性。
% 多无人机航迹规划及任务分配问题
在多无人机航迹规划及任务分配问题中,RRT*-Connect算法的能够在城市密集障碍物静态场景下快速规划出优良的无人机飞行航迹,相较于其他对比算法,其规划的飞行航迹有着更好的性能和鲁棒性;A*算法能够根据实时获取的障碍物信息对无人机飞行航迹进行快速修订,相较于其他对比算法,其生成的飞行航迹与运行时间都有着显著的优越性和鲁棒性;同时的TSAVN算法能够根据所给飞行航程信息与任务信息,在较短时间内生成无人机的收集型侦察任务分配方案,相较于其他对比算法,其生成的任务分配方案以及算法的运行时间都有着显著的优越性与鲁棒性。

\end{abstractzh}
