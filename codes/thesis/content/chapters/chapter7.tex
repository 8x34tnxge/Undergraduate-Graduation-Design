%!TEX root = ../../csuthesis_main.tex
\chapter{总结与展望}

\section{总结}

本文的主要研究成果如下:

\begin{enumerate}[leftmargin=2em]
    \item {对无人机航迹规划及任务调度问题进行了具体的描述与内容分析,在基于边缘计算架构将其分解为面向边缘计算的任务计算资源调度优化问题、多无人机的航迹规划及任务分配问题等两个子问题,基于合理假设为这两个子问题分别构建任务计算资源的调度优化模型、多无人机航迹规划及任务分配模型,为系统研究这两个问题提供了研究基础;}

    \item {针对任务计算资源的调度优化问题,考虑静态和动态两种不同的场景,设计了面向边缘计算的任务资源调度算法ASA、FD,在性能测试中,通过与SA算法的比较,表明ASA算法在任务完成时间、设备空闲时间等目标值上更优,并且与SA算法相比运行时间有所减少,而FD算法虽然得到的解的质量较差,但是其运行时间非常短,能够满足动态场景下的计算资源调度问题的实时性要求;}

    \item {针对多无人机的任务分配及轨迹规划问题,设计了基于RRT*-Connect的静态场景下的无人机航迹规划算法、基于A*的动态场景下的无人机航迹规划算法和基于TSAVN的多无人机任务调度分配算法。在算法性能测试方面,将其与多种算法进行对比,验证了所提算法在求解相应问题时的有效性和优越性。}
\end{enumerate}

\section{展望}

本文使用的多种算法虽然系统地解决了无人机航迹规划及任务调度问题的两个子问题,取得了较好的成功,但在研究过程中还存在着一些不足,
未来需要对本文的研究问题和所用算法进行进一步的优化,主要的展望如下:

\begin{enumerate}[leftmargin=2em]
    \item {无人机实时避障算法中,设置的位置障碍物为固定的,且无人机传感器能够检测到,但实际环境中可能存在着细小的障碍物,例如电线、未知无人机等物体,对于这些物体,由于其可能因为动态性或无法侦测性,可能导致无人机出现损毁等现象,因此有必要进一步研究无人机对于动态或细小的障碍物的动态场景下的航迹规划算法;}

    \item {在任务计算资源调度优化问题中,为简化问题,本文假设无人机与无人机、无人机与服务器的通信延迟、计算型任务在无人机和服务器上计算所需要的时间是可以通过例如神经网络、支持向量机等模型进行预测的,但并没有进行深入研究及实现,且该问题对任务计算资源调度优化问题中起着非常大的影响作用,因此有必要进一步研究无人机与无人机、无人机与服务器的通信延迟、计算型任务在无人机和服务器上计算所需要的时间的预测方法;}

    \item {本文将无人机航迹规划及任务调度问题分解为了两个子问题,但并未将两个子问题合并后进行求解,因此本文的研究成果离实际应用还有一定距离。}
\end{enumerate}

\newpage
