%!TEX root = ../csuthesis_main.tex
\keywordsen{Multi-UAV\ \ Edge Computing\ \ Task Scheduling\ \ Trajectory Planning\ \ Resource Scheduling\ \ Intelligent Optimization Algorithm}
\begin{abstracten}

In recent years, with the continuous development of UAV technology, UAVs have been widely used in both military and civilian fields. In order to promote the application of UAVs in urban environments, this thesis investigates the UAV task scheduling and dynamic trajectory planning problem. Based on a detailed description and analysis of the problem, it is decomposed into two sub-problems based on edge computing architecture: the task computing resource scheduling problem, the multi-UAV task allocation and trajectory planning problem, and the single-UAV real-time obstacle avoidance problem. The corresponding mathematical models are established, and the above sub-problems are solved by designing the static mission computing resource scheduling algorithm based on ASA algorithm, the dynamic mission computing resource scheduling algorithm based on FD algorithm, the UAV trajectory planning algorithm in static scenes based on RRT*-Connect algorithm, the UAV trajectory planning algorithm in dynamic scenes based on A* algorithm and the multi-UAV mission allocation algorithm based on TSAVN algorithm algorithm based on A* algorithm. Finally, the corresponding simulation experiments are designed for each subproblem to verify the effectiveness and superiority of the proposed algorithms in solving each subproblem of UAV task scheduling and dynamic trajectory planning considering edge computing.

The simulation experimental results show that 
% 任务计算资源调度问题
in the task computational resource scheduling problem, the ASA algorithm is able to generate a better computational resource scheduling scheme resource information, considering the transmission delay. At the same time, its generated computational resource scheduling scheme and the running time of the algorithm are significantly superior and robust compared to other comparative algorithms. The FD algorithm, on the other hand, generates a poorer computational resource scheduling scheme compared to other comparative algorithms, but its running time can meet the real-time performance required in the problem scenario.
% 多无人机航迹规划及任务分配问题
In the multi-UAV mission assignment and trajectory planning problem, the RRT*-Connect algorithm is able to quickly plan excellent UAV flight trajectories in urban dense obstacle scenarios with better performance and robustness compared to other comparative algorithms; the A* algorithm can quickly revise the UAV flight trajectory based on the obstacle information acquired in real time, which has significant superiority and robustness in terms of generated flight trajectory and running time compared to other comparative algorithms;meanwhile, the TSAVN algorithm is able to generate UAV collection-based reconnaissance mission assignment schemes in a short time based on the given flight distance information and mission information, which has significant superiority and robustness compared to other comparative algorithms in terms of the generated mission assignment schemes and the running time of the algorithm.

\end{abstracten}
